\section*{Introducción}

\begin{frame}\frametitle{¿Cuál es el objetivo?}
\begin{columns}[T]
\begin{column}{.55\textwidth}
    \begin{block}{Necesidades}
        \begin{itemize}
            \item Mantener copias de seguridad
            \item Gestionar versiones
            \item Hacer los proyectos continuables
            \item Hacer el desarrollo del proyecto visible
            \item Crear discusión y colaboración
        \end{itemize}
    \end{block}
\end{column}

\begin{column}{.4\textwidth}
    \begin{block}{Herramientas}
        \begin{itemize}
            \item Sistema de control de versiones: Git
            \item Licencias
            \item Proyectos \textit{Open Source}
        \end{itemize}
    \end{block}
\end{column}
\end{columns}
\end{frame}

\begin{frame}\frametitle{¿Git para matemáticos?}
\begin{center}
\pause
\includegraphics[width=2cm]{images/linux-logo.jpg}

\pause
\includegraphics[width=2cm]{images/r-logo.jpg}
\includegraphics[width=2cm]{images/matlab-logo.jpg}
\includegraphics[width=2cm]{images/python-logo.jpg}
\includegraphics[width=2cm]{images/julia-logo.png}

\pause
\includegraphics[width=2cm]{images/ctan-logo.png}
\includegraphics[width=2cm]{images/gap-logo.png}
\end{center}
\end{frame}

\begin{frame}\frametitle{¿Qué puedo hacer yo por vosotros?}
\begin{itemize}
    \item Esta pequeña introducción

    \item Aportaros asistencia técnica

    \item Contribuir en los repositorios que creéis

\pause
\vfill

    \item Y además tengo una pequeña sorpresa preparada para el final del webinar
    que os hará la vida más fácil.
\end{itemize}
\end{frame}

